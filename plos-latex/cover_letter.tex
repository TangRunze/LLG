%%%%%%%%%%%%%%%%%%%%%%%%%%%%%%%%%%%%%%%%%
% Professional Formal Letter
% LaTeX Template
% Version 1.0 (28/12/13)
%
% This template has been downloaded from:
% http://www.LaTeXTemplates.com
%
% Original author:
% Brian Moses (http://www.ms.uky.edu/~math/Resources/Templates/LaTeX/)
% with extensive modifications by Vel (vel@latextemplates.com)
%
% License:
% CC BY-NC-SA 3.0 (http://creativecommons.org/licenses/by-nc-sa/3.0/)
%
%%%%%%%%%%%%%%%%%%%%%%%%%%%%%%%%%%%%%%%%%

%----------------------------------------------------------------------------------------
%   PACKAGES AND OTHER DOCUMENT CONFIGURATIONS
%----------------------------------------------------------------------------------------

\documentclass[11pt]{letter} % Specify the font size (10pt, 11pt and 12pt) and paper size (letterpaper, a4paper, etc)

\usepackage{graphicx} % Required for including pictures
\usepackage{microtype} % Improves typography
\usepackage{gfsdidot} % Use the GFS Didot font: http://www.tug.dk/FontCatalogue/gfsdidot/
\usepackage[T1]{fontenc} % Required for accented characters
\usepackage{color}



\newcommand{\red}[1]{{\color{red} #1}}

% Create a new command for the horizontal rule in the document which allows thickness specification
\makeatletter
\def\vhrulefill#1{\leavevmode\leaders\hrule\@height#1\hfill \kern\z@}
\makeatother

%----------------------------------------------------------------------------------------
%   DOCUMENT MARGINS
%----------------------------------------------------------------------------------------

% \textwidth 6.75in
% \textheight 9.25in
% \oddsidemargin -0.25in
% \evensidemargin -0in
% \topmargin -1in
% \longindentation 0.50\textwidth
% \parindent 0.4in

\usepackage{fullpage}

\renewcommand{\familydefault}{\sfdefault}

%----------------------------------------------------------------------------------------
%   SENDER INFORMATION
%----------------------------------------------------------------------------------------

\def\Who{Daniel L. Sussman} % Your name
\def\What{, PhD} % Your title
\def\Where{} % Your department/institution
\def\Address{111 Cummington Mall} % Your address
\def\CityZip{Boston MA, 02215} % Your city, zip code, country, etc
\def\Email{sussman@bu.edu} % Your email address
\def\TEL{Phone: (443) 604-5241} % Your phone number
% \def\URL{URL: http://www.people.fas.harvard.edu/~daniellsussman/cv.html} % Your URL

%----------------------------------------------------------------------------------------
%   HEADER AND FROM ADDRESS STRUCTURE
%----------------------------------------------------------------------------------------

\address{
\hspace{5in} \hfill \includegraphics[width=1.3in]{boston_univ_rgb.png}
 % Include the logo of your institution
 % Position of the institution logo, increase to move left, decrease to move right
% \vskip .1in~\\ % Position of the text in relation to the institution logo, increase to move down, decrease to move up
\hspace{0in}\textbf{Boston University} College of Arts and Sciences \hfill ~\\[0.05in] % First line of institution name, adjust hspace if your logo is wide
\hspace{0in}Department of Mathematics \& Statistics \hfill \normalsize % Second line of institution name, adjust hspace if your logo is wide
\makebox[0ex][r]{\bf \Who \What }\hspace{0in} % Print your name and title with a little whitespace to the right
~\\[-0.11in] % Reduce the whitespace above the horizontal rule
\hspace{0in}\vhrulefill{1pt} \\ % Horizontal rule, adjust hspace if your logo is wide and \vhrulefill for the thickness of the rule
\hspace{\fill}\parbox[t]{3.08in}{ % Create a box for your details underneath the horizontal rule on the right
\footnotesize % Use a smaller font size for the details
% \Who \\ \em % Your name, all text after this will be italicized
% \Where\\ % Your department
\Address\\ % Your address
\CityZip\\ % Your city and zip code
% \TEL\\ % Your phone number
\Email\\ % Your email address
% \URL % Your URL
}
\hspace{-1.4in} % Horizontal position of this block, increase to move left, decrease to move right
\vspace{-1in} % Move the letter content up for a more compact look
}

%----------------------------------------------------------------------------------------
%   TO ADDRESS STRUCTURE
%----------------------------------------------------------------------------------------

\def\opening#1{\thispagestyle{empty}
{\centering\fromaddress \vspace{0.75in} \\ % Print the header and from address here, add whitespace to move date down
} % Print today's date, remove \today to not display it
{\raggedright \toname \\ \toaddress \par} % Print the to name and address
\vspace{0.3in} % White space after the to address
\noindent #1 % Print the opening line
% Uncomment the 4 lines below to print a footnote with custom text
%\def\thefootnote{}
%\def\footnoterule{\hrule}
%\footnotetext{\hspace*{\fill}{\footnotesize\em Footnote text}}
%\def\thefootnote{\arabic{footnote}}
}

%----------------------------------------------------------------------------------------
%   SIGNATURE STRUCTURE
%----------------------------------------------------------------------------------------

\signature{\Who,\\ on behalf of my co-authors.} % The signature is a combination of your name and title

\long\def\closing#1{
\vspace{0.1in} % Some whitespace after the letter content and before the signature
\noindent % Stop paragraph indentation
\hspace*{\longindentation} % Move the signature right
\parbox{\indentedwidth}{\raggedright
#1 % Print the signature text
\vskip 0.15in % Whitespace between the signature text and your name
\fromsig}} % Print your name and title

%----------------------------------------------------------------------------------------

\begin{document}

\begin{letter}{%Chair, Faculty Search Committee\\
\today\\
\vspace{11pt}
Tilmann Gneiting and Editorial Board \\
The Annals of Applied Statistics\\
Institute of Mathematical Statistics
}


\opening{Dear STilmann Gneiting and Editorial Board of AoAS,}

We submit to you our manuscript titled ``Connectome Smoothing via Low-rank Approximations'' for publication as a regular paper.

As neuroscientists, computational biologists, and social scientists measure increasingly large networks, tools to accurately estimate the mean network are increasingly important.
When confronting high-dimensional data, even estimating the mean can require careful statistical thinking, and with graph data the number of parameters for the mean scales quadratically with the number of vertices.
% and so using the standard sample mean of the adjacency matrices may lead to excessive variance.
The performance of the standard sample mean can be especially poor in cases with small samples sizes and large number of vertices.
Additionally, even if the number of networks collected in a study is large, researchers often seek to estimate the mean for smaller subsets of interest.
The challenge is especially acute for connectomics, the study of brain networks, where accurate measurements of connectivity at the micro- and meso-scales are still in development.
Our goal is to propose a method which will improve the performance for this estimation task in these cases.

We proposed a novel method motivated by a low-rank statistical model for random networks.
We show that this end-to-end method for estimating the mean of a collection of graphs yields improvements over using standard methods.
This is especially evident when considering small cohorts with large numbers of vertices, as is increasingly common practice in clinical studies.
% These methods offer substantial improvements in the increasingly common setting where the number of vertices in the graph is large but the number of graphs in the sample is relatively small.
We also illustrate how the low-rank nature of of our method illuminates relationships between connectivity and anatomical structure that are not as evident from more standard techniques.
We believe scientists in a variety of fields will find these methods useful when confronted with collections graphs and our associated analysis will help practitioners decide on the best estimator for their setting.

Note, this is a revision of a manuscript, entitled ``Law of Large Graphs'', previously submitted to another journal.
We hope you enjoy reading the manuscript and look forward to your feedback.

\closing{Best,}


\end{letter}
\end{document}